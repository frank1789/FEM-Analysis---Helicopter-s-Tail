\chapter{Complete Model}
Now describe the procedure to attach the tail rotor to two structure: the truss model and shell model, respectively.
In both base model we add a part that realize the nodes that represent rigid connection between the tail structure and tail rotor.
After call the macro to realize the component, realized connection by rigid elements between these and component previous realized.

\section{Truss model with rotor tail}
As previously introduced, we use the base model of the trellis structure, which serves as a base for attaching the tail rotor developed in detail in the chapter.
Two nodes are placed at the end of the tail and two new raised nodes of a certain offset are created that will act as a base for the bearing attachment.
Once the basic nodes are made, the macro \ref{lst:rotor} is used to make the rotor model.
To allow rotation of the tail rotor shaft, the rigid element \textsc{mpc184} with (keyopt, 6) was used in such a way as to produce joint revolutions placed in their respective nodes of attack.
This result allows the shaft to rotate along its axis and simultaneously limiting other movements.
Finally, the two structures are connected by rigid elements of the \textsc{mpc184} type, the final result is shown in the figure

%\listingautocaption[language=apdl-modified, label={lst:ShellModeldiskRun},firstline=170,lastline=236]{ShellModeldiskRun.txt}



%firstline=300,lastline=500
\section{Shell model with rotor Tail}

